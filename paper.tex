\section{Ausgangssituation}\label{ausgangssituation}

Softwareentwicklung für den Browser ist ohne Zweifel eine komplexe
Angelegenheit. Die Gründe für diesen Umstand sind so vielfältig, dass
darüber eine ganze Forschungsarbeit geschrieben werden könnte. Je nach
Blickwinkel könnte man von der technischen Seite argumentieren und
beispielsweise Designschwächen von JavaScript oder die unflexiblen
Browserengines als Bremsklotz der Webentwicklung benennen.{[}1{]} Oder
man könnte die organisatorische Seite betrachten und die vielen
beteiligten Organisationen bzw. Unternehmen und die daraus
resultierenden langsamen Innovationszyklen als Problem benennen.{[}2{]}
Oder aber man betrachtet die historische Entwicklung, die dazu geführt
hat, dass die technische Entwicklung der rasanten Evolution im Internet
nicht Schritt gehalten hat. Die Zeiten statischer HTML/CSS Seiten auf
einfachen Desktopgeräten ist mithin noch gar nicht so lange her.

Dennoch tragen alle diese Facetten dazu bei, dass die Entwicklung und
Wartung von komplexen Webapplikationen, wie sie heute Standard sind, mit
enormen Zeit- und Geldaufwand verbunden sind. Die Anzahl der Frameworks,
Werkzeuge und Bibliotheken mit Javascript als Zielsprache ist
unüberblickbar geworden und wandelt sich in einer Geschwindigkeit, die
vielen Entwicklern Schwierigkeiten bereitet.\footnote{https://hackernoon.com/how-it-feels-to-learn-javascript-in-2016-d3a717dd577f}
Aber auch die andere Seite, die Nutzer der Webdienste, bekommen diese
Probleme spüren. Frederic Filloux zeigte in einem Blogpost, dass sich in
einem 4667 Buchstaben langen Zeitungsartikel der britischen Zeitung
``The Guardian'' 485527 Buchstaben an Quelltext verstecken.{[}3{]}

Um diesen Entwicklungen, die viel Zeit und Geld kosten, etwas
entgegenzusetzen,

https://extensiblewebmanifesto.org/

\section{Zielsetzung}\label{zielsetzung}

Ziel dieser Arbeit ist die systematische Erfassung der neuen
Technologien sowie eine praktische Erläuterung möglicher
Architekturmodelle.

\section{Vorgehensweise}\label{vorgehensweise}

Den Anfang dieser Arbeit soll eine Analyse der aktuellen Situation der
Frontend Entwicklung aufzeigen. In ihr sollen aktuelle
Architekturmodelle analysiert werden. Außerdem soll dieser Teil vor
Augen führen, warum es bisher nur mit zusätzlichen Schichten der
Abstraktion möglich war einen modularen Aufbau von Web Apps zu
ermöglichen. In diesem Teil sollen auch die Probleme die der Entwicklung
und Wartung dieser Systeme aufzeigen.

Im nächsten Abschnitt der Arbeit sollen die

\hypertarget{refs}{}
\hypertarget{ref-Katz2013}{}
{[}1{]} Y. Katz, ``Extend the web forward,'' 2013.

\hypertarget{ref-Walton2016}{}
{[}2{]} P. Walton, ``Houdini: Maybe the most exciting development in css
you've never heard of,'' 2016.

\hypertarget{ref-Filloux2016}{}
{[}3{]} F. Filloux, ``Bloated html, the best and the worse,'' 2016.
