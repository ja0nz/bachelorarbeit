\section{Ausgangssituation}\label{ausgangssituation}

Softwareentwicklung für den Browser ist ohne Zweifel eine komplexe
Angelegenheit. Die Gründe für diesen Umstand sind so vielfältig, dass
darüber eine ganze Forschungsarbeit geschrieben werden könnte. Je nach
Blickwinkel könnte man von der technischen Seite argumentieren und
beispielsweise Designschwächen von JavaScript oder die unflexiblen
Browserplattformen als Bremsklotz der Webentwicklung benennen.{[}1{]}
Oder man könnte die organisatorische Seite betrachten und die vielen
beteiligten Organisationen benennen, die für Standardisierungsprozesse
im Internet zuständig sind.{[}2{]} Oder aber man nimmt die historische
Entwicklung in Augenschein, die dazu geführt hat, dass die technische
Entwicklung der rasanten Evolution im Internet nicht Schritt gehalten
hat. Die Zeiten statischer HTML/CSS Seiten auf einfachen Desktopgeräten
ist mithin noch gar nicht so lange her.

Dennoch tragen alle diese Facetten dazu bei, dass die Entwicklung und
Wartung von komplexen Webapplikationen, wie sie heute Standard sind, mit
enormen Zeit- und Geldaufwand verbunden sind. Die Anzahl der Frameworks,
Werkzeuge und Bibliotheken mit Javascript als Zielsprache ist nahezu
unüberblickbar und wandelt sich in einer Geschwindigkeit, die vielen
Entwicklern Schwierigkeiten bereitet.{[}3{]} Auch die Konsumenten und
Nutzern der Webdiensten bekommen diese Probleme zumindest teilweise
spüren. Einerseits spürbar in Form von ``Downtimes'' der Services.
Andererseits subtil wie etwa durch lange Ladezeiten wegen aufgeblähtem
Quellcode. Frederic Filloux zeigte in einem Blogpost anschaulich, dass
sich in einem Zeitungsartikel der britischen Zeitung ``The Guardian'' zu
jedem lesbaren Buchstaben des Artikels über 100 Zeichen Code
addieren.{[}4{]}

Aus diesen Grund wurde schon 2013 das ``Extensible Web Manifesto''
proklamiert. Darin wird, kurz gefasst, eine Öffnung der Webplattformen
avisiert, um Webprogrammierung teilweise von Browserstandards zu
entkoppeln.\footnote{https://extensiblewebmanifesto.org/} Drei Jahre
später sind diese Ziele in W3C Spezifikationen konkretisiert worden.
Einige dieser neuen Standards werden bereits nativ in (einigen) Browsern
unterstützt.

\section{Zielsetzung}\label{zielsetzung}

Viele populäre Frameworks für den Browser als Zielplattform versprechen
dem Entwickler eine bessere Kontrolle über die Webseite. Der Grund dafür
liegt unter anderem darin, dass es bisher nicht möglich war einzelne
Browserelemente in ihrem Aussehen und Verhalten zu isolieren. Selbst
kleinste Änderungen können daher die Funktionalität des komplexen
Gesamtsystems beeinträchtigen. Mit der neuen W3C Spezifikation ``Web
Components'', die mehrere Substandards unter sich vereint, soll die
Modularität nun auch nativ im Browser unterstützt werden.\footnote{https://www.w3.org/standards/techs/components}
Modularität als Designpattern ist in der IT schon länger Standard um
komplexe Systeme beherrschbar zu gestalten und auch zukünftige
Unsicherheiten abzuwägen.{[}5, p. 1{]} Die Unsicherheit, die Frameworks
immer anhaftet, ist die Lebensdauer derselben.

Web Components ist der Entwickler in der Lage, eigene HTML Elemente zu
definieren und diese per Templates zu exportieren oder zu importieren.
Darüber hinaus können sie JavaScript und CSS Funktionalität enthalten
und diese vom Rest der Webseite abzukapseln. Ziel dieser Arbeit ist die
systematische Erfassung und Risikoabwägung der neuen Technologien sowie
eine praktische Erläuterung möglicher Architekturmodelle. Bisher gibt es
keine einheitlichen ``best practices'' wie denn die Komponenten
umzusetzen sind, obwohl diese Technologien bereits länger durch
Polyfills genutzt werden können. Manch ein Entwickler fürchtet schon
eine Flut von schlecht gebauten Komponenten, die wiederum neue Probleme
schaffen könnten.{[}6{]}

\section{Vorgehensweise}\label{vorgehensweise}

Den Anfang dieser Arbeit soll eine Analyse der aktuellen Situation der
Frontend Entwicklung aufzeigen. Dort soll auf die Problematik der
bisherigen Architekturmodelle und die Probleme mit monolithischen
Frameworks eingegangen werden. Außerdem soll dieser Teil vor Augen
führen, warum es bisher nur mit zusätzlicher Abstraktionsebenen möglich
war einen modularen Aufbau von Web Applikationen zu ermöglichen.

Im nächsten Abschnitt der Arbeit sollen die neuen Standards
aufgeschlüsselt, zugänglich gemacht und auf Anwendungsmöglichkeiten
untersucht werden. Explorativ sollen gute Bespiele offengelegt werden
und mögliche Einbettungen ins Gesamtsystem diskutiert werden. Auch die
Probleme, wie sie beispielsweise bei alten Browsern auftreten können,
sollen hier behandelt werden.

Der letzte Abschnitt soll das Thema noch mit einer Metaperspektive
abrunden, in der auch die neuen low-level CSS Standards Houdini als Teil
des Gesamtsystems betrachtet werden sollen.\footnote{https://drafts.css-houdini.org/}

\hypertarget{refs}{}
\hypertarget{ref-Katz2013}{}
{[}1{]} Y. Katz, ``Extend the web forward,'' 2013.

\hypertarget{ref-Walton2016}{}
{[}2{]} P. Walton, ``Houdini: Maybe the most exciting development in css
you've never heard of,'' 2016.

\hypertarget{ref-Berner2016}{}
{[}3{]} D. Berner, ``Not an imposter: Fighting front-end fatigue,''
2016.

\hypertarget{ref-Filloux2016}{}
{[}4{]} F. Filloux, ``Bloated html, the best and the worse,'' 2016.

\hypertarget{ref-Baldwin2006}{}
{[}5{]} C. Y. Baldwin and K. B. Clark, ``Modularity in the design of
complex engineering systems,'' in \emph{Complex engineered systems:
Science meets technology}, D. Braha, A. A. Minai, and Y. Bar-Yam, Eds.
Berlin, Heidelberg: Springer Berlin Heidelberg, 2006, pp. 175--205.

\hypertarget{ref-Keith2014}{}
{[}6{]} J. Keith, ``Web components,'' 2014.
