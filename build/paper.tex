\documentclass[ngerman,]{../assets/latex/ieee}
\usepackage{lmodern}
\usepackage{amssymb,amsmath}
\usepackage{ifxetex,ifluatex}
\usepackage{fixltx2e} % provides \textsubscript
\ifnum 0\ifxetex 1\fi\ifluatex 1\fi=0 % if pdftex
  \usepackage[T1]{fontenc}
  \usepackage[utf8]{inputenc}
\else % if luatex or xelatex
  \ifxetex
    \usepackage{mathspec}
  \else
    \usepackage{fontspec}
  \fi
  \defaultfontfeatures{Ligatures=TeX,Scale=MatchLowercase}
\fi
% use upquote if available, for straight quotes in verbatim environments
\IfFileExists{upquote.sty}{\usepackage{upquote}}{}
% use microtype if available
\IfFileExists{microtype.sty}{%
\usepackage{microtype}
\UseMicrotypeSet[protrusion]{basicmath} % disable protrusion for tt fonts
}{}
\usepackage[unicode=true]{hyperref}
\hypersetup{
            pdftitle={Browsernative Microservices},
            pdfauthor={Jan Peteler, FH Würzburg-Schweinfurt, jan.peteler@student.fhws.de},
            pdfborder={0 0 0},
            breaklinks=true}
\urlstyle{same}  % don't use monospace font for urls
\ifnum 0\ifxetex 1\fi\ifluatex 1\fi=0 % if pdftex
  \usepackage[shorthands=off,main=ngerman]{babel}
\else
  \usepackage{polyglossia}
  \setmainlanguage[]{german}
\fi
\usepackage{color}
\usepackage{fancyvrb}
\newcommand{\VerbBar}{|}
\newcommand{\VERB}{\Verb[commandchars=\\\{\}]}
\DefineVerbatimEnvironment{Highlighting}{Verbatim}{commandchars=\\\{\}}
% Add ',fontsize=\small' for more characters per line
\newenvironment{Shaded}{}{}
\newcommand{\KeywordTok}[1]{\textcolor[rgb]{0.00,0.44,0.13}{\textbf{{#1}}}}
\newcommand{\DataTypeTok}[1]{\textcolor[rgb]{0.56,0.13,0.00}{{#1}}}
\newcommand{\DecValTok}[1]{\textcolor[rgb]{0.25,0.63,0.44}{{#1}}}
\newcommand{\BaseNTok}[1]{\textcolor[rgb]{0.25,0.63,0.44}{{#1}}}
\newcommand{\FloatTok}[1]{\textcolor[rgb]{0.25,0.63,0.44}{{#1}}}
\newcommand{\ConstantTok}[1]{\textcolor[rgb]{0.53,0.00,0.00}{{#1}}}
\newcommand{\CharTok}[1]{\textcolor[rgb]{0.25,0.44,0.63}{{#1}}}
\newcommand{\SpecialCharTok}[1]{\textcolor[rgb]{0.25,0.44,0.63}{{#1}}}
\newcommand{\StringTok}[1]{\textcolor[rgb]{0.25,0.44,0.63}{{#1}}}
\newcommand{\VerbatimStringTok}[1]{\textcolor[rgb]{0.25,0.44,0.63}{{#1}}}
\newcommand{\SpecialStringTok}[1]{\textcolor[rgb]{0.73,0.40,0.53}{{#1}}}
\newcommand{\ImportTok}[1]{{#1}}
\newcommand{\CommentTok}[1]{\textcolor[rgb]{0.38,0.63,0.69}{\textit{{#1}}}}
\newcommand{\DocumentationTok}[1]{\textcolor[rgb]{0.73,0.13,0.13}{\textit{{#1}}}}
\newcommand{\AnnotationTok}[1]{\textcolor[rgb]{0.38,0.63,0.69}{\textbf{\textit{{#1}}}}}
\newcommand{\CommentVarTok}[1]{\textcolor[rgb]{0.38,0.63,0.69}{\textbf{\textit{{#1}}}}}
\newcommand{\OtherTok}[1]{\textcolor[rgb]{0.00,0.44,0.13}{{#1}}}
\newcommand{\FunctionTok}[1]{\textcolor[rgb]{0.02,0.16,0.49}{{#1}}}
\newcommand{\VariableTok}[1]{\textcolor[rgb]{0.10,0.09,0.49}{{#1}}}
\newcommand{\ControlFlowTok}[1]{\textcolor[rgb]{0.00,0.44,0.13}{\textbf{{#1}}}}
\newcommand{\OperatorTok}[1]{\textcolor[rgb]{0.40,0.40,0.40}{{#1}}}
\newcommand{\BuiltInTok}[1]{{#1}}
\newcommand{\ExtensionTok}[1]{{#1}}
\newcommand{\PreprocessorTok}[1]{\textcolor[rgb]{0.74,0.48,0.00}{{#1}}}
\newcommand{\AttributeTok}[1]{\textcolor[rgb]{0.49,0.56,0.16}{{#1}}}
\newcommand{\RegionMarkerTok}[1]{{#1}}
\newcommand{\InformationTok}[1]{\textcolor[rgb]{0.38,0.63,0.69}{\textbf{\textit{{#1}}}}}
\newcommand{\WarningTok}[1]{\textcolor[rgb]{0.38,0.63,0.69}{\textbf{\textit{{#1}}}}}
\newcommand{\AlertTok}[1]{\textcolor[rgb]{1.00,0.00,0.00}{\textbf{{#1}}}}
\newcommand{\ErrorTok}[1]{\textcolor[rgb]{1.00,0.00,0.00}{\textbf{{#1}}}}
\newcommand{\NormalTok}[1]{{#1}}
\IfFileExists{parskip.sty}{%
\usepackage{parskip}
}{% else
\setlength{\parindent}{0pt}
\setlength{\parskip}{6pt plus 2pt minus 1pt}
}
\setlength{\emergencystretch}{3em}  % prevent overfull lines
\providecommand{\tightlist}{%
  \setlength{\itemsep}{0pt}\setlength{\parskip}{0pt}}
\setcounter{secnumdepth}{0}
% Redefines (sub)paragraphs to behave more like sections
\ifx\paragraph\undefined\else
\let\oldparagraph\paragraph
\renewcommand{\paragraph}[1]{\oldparagraph{#1}\mbox{}}
\fi
\ifx\subparagraph\undefined\else
\let\oldsubparagraph\subparagraph
\renewcommand{\subparagraph}[1]{\oldsubparagraph{#1}\mbox{}}
\fi

% set default figure placement to htbp
\makeatletter
\def\fps@figure{htbp}
\makeatother


\title{Browsernative Microservices}
\providecommand{\subtitle}[1]{}
\subtitle{Modulare Webarchitekturen durch neue Browserstandards entwickeln}
\author{Jan Peteler, FH Würzburg-Schweinfurt, jan.peteler@student.fhws.de}
\date{Januar 2017}

\begin{document}
\maketitle

\section{Ausgangssituation}\label{ausgangssituation}

en der Browserplattform erläutern.{[}1{]} Oder man könnte die
organisatorische Seite betrachten und die Schwerfälligkeit von
Standardisierungsprozessen anprangern.{[}2{]} Oder aber man nimmt die
historische Entwicklung in Augenschein. Die Zeiten statischer HTML/CSS
Seiten auf Desktopgeräten ist mithin noch nicht lange her.

cript als Zielsprache betrachtet, die vielen Entwicklern Schwierigkeiten
bereitet.{[}3{]}

nftige Unsicherheiten abzuwägen.{[}4, S. 1--3{]} Die Unsicherheit, die
Frameworks immer anhaftet, ist die Lebensdauer derselben.

Mit \emph{Web Components} ist der Entwickler in der Lage, eigene HTML
Elemente zu definieren und diese per Templates zu ex- \& importieren.
Darüber hinaus können sie JavaScript und CSS Funktionalität enthalten
und diese vom Rest der Webseite abzukapseln. Ziel dieser Arbeit ist die
systematische Erfassung und Risikoabwägung der neuen Technologien sowie
eine praktische Erläuterung möglicher Architekturmodelle. Bisher gibt es
keine einheitliche Bestpractice, wie denn die Komponenten umzusetzen
sind, obwohl diese Technologien bereits länger durch Polyfills genutzt
werden können. Manch ein Entwickler fürchtet schon eine Flut von
schlecht gebauten Komponenten, die wiederum neue Probleme schaffen
könnten.{[}5{]}

\section{Vorgehensweise}\label{vorgehensweise}

Den Anfang dieser Arbeit soll eine Analyse der aktuellen Situation der
Frontend Entwicklung aufzeigen. Dort soll auf die Problematik der
bisherigen Architekturmodelle und die Probleme mit monolithischen
Frameworks eingegangen werden. Außerdem soll dieser Teil vor Augen
führen, warum es bisher nur mit zusätzlicher Abstraktionsebenen möglich
war einen modularen Aufbau von Web Applikationen zu ermöglichen.

Im nächsten Abschnitt der Arbeit sollen die neuen Standards
aufgeschlüsselt, zugänglich gemacht und auf Anwendungsmöglichkeiten
untersucht werden. Explorativ sollen Bestpractices offengelegt werden
und mögliche Einbettungen ins Gesamtsystem diskutiert werden. Auch die
Probleme, wie sie beispielsweise bei alten Browsern auftreten können,
sollen hier behandelt werden.\\
So we are

\begin{quote}
This is a blockquote
\end{quote}

´hello world´

\begin{Shaded}
\begin{Highlighting}[]
\NormalTok{Javascript baby}
\end{Highlighting}
\end{Shaded}

\texttt{so\ this\ is\ the\ js\ world}

one more last test

\begin{enumerate}
\def\labelenumi{\arabic{enumi}.}
\tightlist
\item
  This is the list
\end{enumerate}

This is a sub

\begin{enumerate}
\def\labelenumi{\arabic{enumi}.}
\setcounter{enumi}{1}
\tightlist
\item
  Anothoter list
\end{enumerate}

\begin{itemize}
\tightlist
\item
  {[} {]} oohhhhhhh
\end{itemize}

Der letzte Abschnitt soll das Thema mit einer Metaebene abrunden, in der
auch die neuen CSS Standards \emph{Houdini} als Teil des Gesamtsystems
betrachtet werden sollen.\footnote{https://drafts.css-houdini.org/}

\hypertarget{refs}{}
\hypertarget{ref-Katz2013}{}
{[}1{]} Y. Katz, „Extend the Web Forward``, 2013 {[}Online{]}. Verfügbar
unter: \url{http://yehudakatz.com/2013/05/21/extend-the-web-forward/}

\hypertarget{ref-Walton2016}{}
{[}2{]} P. Walton, „Houdini: Maybe The Most Exciting Development In CSS
You've Never Heard Of``, 2016 {[}Online{]}. Verfügbar unter:
\url{https://www.smashingmagazine.com/2016/03/houdini-maybe-the-most-exciting-development-in-css-youve-never-heard-of}

\hypertarget{ref-Berner2016}{}
{[}3{]} D. Berner, „Not An Imposter: Fighting Front-End Fatigue``, 2016
{[}Online{]}. Verfügbar unter:
\url{https://www.smashingmagazine.com/2016/11/not-an-imposter-fighting-front-end-fatigue/}

\hypertarget{ref-Dodson2016}{}
{[}4{]} R. Dodson, „The Case for Custom Elements: Part 2 -- Dev Channel
-- Medium``. 2016 {[}Online{]}. Verfügbar unter:
\href{https://medium.com/dev-channel/the-case-for-custom-elements-part-2-2efe42ce9133\%7B/\#\%7D.gl5nuty2r}{https://medium.com/dev-channel/the-case-for-custom-elements-part-2-2efe42ce9133\{\textbackslash{}\#\}.gl5nuty2r}.
{[}Zugegriffen: 01-Dez-2016{]}

\hypertarget{ref-Keith2014}{}
{[}5{]} J. Keith, „Web Components``, 2014 {[}Online{]}. Verfügbar unter:
\url{https://adactio.com/journal/7431}

\end{document}
